\documentclass[11pt]{article}
\usepackage[cm]{fullpage}
\usepackage{bbold}
\usepackage{setspace}
\usepackage{amsmath}
\usepackage{bbm}
\usepackage{graphicx}
\usepackage{float}
\usepackage{multirow}
\usepackage{url}
\usepackage{algorithmic}
\usepackage{algorithm}
\usepackage{amsthm}
\usepackage{mathtools}
\usepackage{array}
\DeclarePairedDelimiter{\ceil}{\lceil}{\rceil}
\setlength\extrarowheight{0.1cm}

\DeclareGraphicsExtensions{.pdf,.png,.jpg}

\setlength\parindent{0pt}

\begin{document}
\begin{tabular}{|>{\raggedright}m{4.5cm}|m{13cm}|} \hline
\textbf{Woodbury Matrix Identity} & $(A+UCV)^{-1} = A^{-1} - A^{-1}U(C^{-1}+VA^{-1}U)^{-1}VA^{-1}$\\ \hline
\textbf{Sherman Morrison Formula} & Given \textbf{invertible} $A \in R^{n\times n}$ and $u,v\in R^n$ vectors. Then $A+uv^T$ is invertible iff $1+v^TA^{-1}u \neq 0$. Then $(A+uv^T)^{-1} = A^{-1}-\frac{A^{-1}uv^TA^{-1}}{1+v^TA^{-1}u}$\\ \hline
\textbf{Cauchy Schwarz Inequality} & $ |\langle U, V \rangle| \leq \langle U, U\rangle \dot \langle V, V\rangle $\\ \hline
\textbf{Markov Inequality} & $X$ is a non-negative rv and $a>0$ then $P(X \geq a) \leq \frac{E(X)}{a}$\\ \hline
\textbf{Minkowski inequality} & $\|f+g\|_p \leq \|f\|_p + \| g\|_p$\\ \hline
\textbf{Holder Inequality} & Let $p,q\in [1, \infty], 1/p+1/q=1$. Then for all real/complex functions $f,g$,$\|fg|_1 \leq \|f\|_p\|g\|_q$\\ \hline
\textbf{Jensen Inequality} & Given X is a rv and $\phi$ is convex. Then $\phi(E(X)) \leq E(\phi(X))$\\ \hline
\textbf{Johnson Lindenstrauss Property} & A distribution on matrices $S\in R^{k \times n}$ has $(\epsilon, \delta, l)$-JL moment property if $\forall x\in R^n, |x|_2=1$, $E_S(||Sx|^2_2-1|^l)\leq \epsilon^l\delta$\\ \hline
\textbf{Hoeffding Inequality} & Let $X_1,\cdots,X_n$ be independent rv in $[0,1]$. Let $\bar{X} = \frac{1}{n}\sum_i{X_i}$. Then $P(\bar{X}-E(\bar{X})\geq t) \leq \exp(-2nt^2)$ \\ \hline
\textbf{Chebyshev Inequality} & $P(|X-E(X)| \geq a) \leq \frac{Var(X)}{a^2}$\\ \hline
\textbf{Golden Thompson} & For Hermitian matrices A,B $tr \exp(A+B) \leq tr(\exp(A)\exp(B))$\\ \hline
\end{tabular}
\end{document}
